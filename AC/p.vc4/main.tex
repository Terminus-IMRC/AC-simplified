\def \wrkdir {p.vc4/}

\mpart{VideoCore IV 向けの低レベルなライブラリの開発}
\label{pa:vc4}

VideoCore IVは、Raspberry Pi 1のメインのチップであるBroadcom BCM2708、また、
Raspberry Pi 2のメインのチップであるBroadcom BCM2709に内蔵されているGPUである。

VideoCore IVにはQPUというプロセッサが含まれている。
QPUはVideoCore IVに12基あり、1基には4個の物理コアがある。
物理コア1個は4クロックで4個の仮想コアをエミュレートする。
つまり、全体では48個の物理コア、192個の仮想コアがある。

QPUの物理コアは内部にレジスタやALUを持つので、計算処理を行うことができる。
しかし、メモリを読み書きしたり逆数を求めたりするためには
周辺のユニットと通信をしなければならない。
その通信の際にはストールが発生するが、逆に言えば周辺のユニットとの通信以外ではストールは発生しない。
ストールする命令を実行させなければ、QPUの物理コアは1クロックで1つの命令を実行することができる。
QPUは標準で250MHzで動作するので、ストールが発生しない場合、
QPU 12基では1秒あたり$12 \times 10^9$個の命令を実行することができる。

私はQPUの高性能性に着目し、QPUで数値計算プログラムを実行したいと考えた。
よって今回は、アセンブリ言語を新たに定義し、そのアセンブラを作成するとともに、
メモリ確保やハードウェアレジスタへのアクセス等をサポートするライブラリを作成した。

今回作成したアセンブラを\cite{bib:qtc}に示す。
ライブラリは、以下のものを作成した。
\begin{itemize}
\item V3D\footnote{CPU側からVideoCore IVのキャッシュや
	走らせるプログラムや割り込み等を設定するためのレジスタ群。} に
	簡単にアクセスするためのライブラリ (\cite{bib:libvc4v3d}に示す)
\item Control List\footnote{VideoCore IVで
	グラフィックの処理をしたりQPUのプログラムを連続的に実行する時に
	使う、命令群を記したメモリ上のリスト。} を簡単に作成するための
	ライブラリ (\cite{bib:libvc4clist}に示す)
\item Mailbox Interface\footnote{GPU側からもアクセスできるメモリを確保したり
	QPUのプログラムを実行したりするためのインターフェイス。} に
	簡単にアクセスするためのライブラリ (\cite{bib:mailbox}に示す)
\item 上記のライブラリをより抽象的に使えるようにしたり、
	ベクトル演算をサポートするライブラリ (\cite{bib:libvc4vec}に示す)
\end{itemize}
