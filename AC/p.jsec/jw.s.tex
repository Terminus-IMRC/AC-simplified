\def \wrkdir {p.jsec/}

\mpart{魔方陣全解出力アルゴリズムの改良と超並列計算環境でのプログラム実行}
\label{pa:jsec}

第\ref{pa:ngk}\postmpartname{}で
学校のコンピュータ室のシステムを利用して並列計算システムを構築した。
そのシステムで4次魔方陣の全解を
総当たりで探索する並列プログラムを実行したところ、
一晩かけても実行が終了しなかった。
その後、T2K-Tsukuba\footnote{筑波大学に設置されていた
スーパーコンピュータ。}学際利用プログラムに申し込み、
そのプログラムをT2K-Tsukubaの512CPUコア上で走らせた。
しかし、最大連続実行可能時間である24時間をかけても実行は終了しなかった。

今回は枝刈り法の改良やより効率の良いアルゴリズムの開発、また、
プログラムの並列化を行った。
その結果、一般的なコンピュータやT2K-Tsukubaで4次魔方陣と5次魔方陣の全解を
探索することができた。
また、T2K-Tsukuba でのプログラム並列実行において最も早く実行完了したのは、
問題を分割する細かさと問題配布に要する通信時間の
バランスが取れている場合であることを見出した。

本研究を第11回高校生科学技術チャレンジに応募したところ、
1次審査で佳作という賞を頂いた。
また、本研究を第3回つくば科学研究コンテスト兼茨城県高校生科学研究発表会で
ポスター発表したところ、審査員奨励賞という賞を頂いた。

その後、上記の、高校生がスーパーコンピュータを使用し、さらに研究成果をあげたことに、
筑波大学計算科学研究センターの広報の方が注目してくださり、
プレスリリースを発行してくださることになった。
広報の方から取材を受けた後、プレスリリースは平成26年2月28日に
発行された(\cite{bib:ccs-pr})。
その後、記者会見を開き、上記の研究は
読売新聞と朝日新聞と茨城新聞と日経新聞の紙面に掲載されたほか、
様々なニュースサイトに掲載された。
