\mmpart{RN-42を用いた方法} \label{sc:usb2bt2}
BluetoothモジュールであるRN-42には、
UART経由で受信したデータをそのままBluetoothの接続先に送信する機能がある。
今回は、RN-42を用いてUSBキーボードをBluetoothに変換した。

マイコンボードとしてはArduino Unoを用いた。
また、マイコンボードとUSBキーボードの接続にはUSB Host Shield 2.0を用いた。

今回はキーマッピングを実装しようと考えた。
ここでのキーマッピングとは、キーボードのキーレイアウトに合わせた入力を
可能にすることと、コンシューマキー等を設定することである。

キーマッピングにはLinuxの標準的なキーマッピングシステムである
kbdを利用することを考えた。
kdbには様々なキーボードのキーレイアウトを記した
キーマッピングファイルが含まれている。
これにより、様々なキーボードのキーレイアウトを新たに定義する必要がなくなり、また、
キーマッピングを適宜編集しやすくなった。

しかし、ArduinoはUSB Host Shield 2.0からHID usage IDを受信し、
RN-42にはHID usage IDを使った形式でデータを送信するので、
キーマッピングのテーブルは変換前と後ともにHID usage IDベースである必要がある。
よって、次の3段階の手順によって、kbdに含まれているキーマッピングファイル中の
テーブルを、HID usage IDをHID usage IDにマップする形式に変換することにした。
\begin{enumerate}
 \item キーコードをasciiコードにマップする
  キーマッピングのテーブルを作成する \label{enum:usb2bt2-mapping-1}
 \item HID usage IDをasciiコードにマップする
  キーマッピングのテーブルを作成する \label{enum:usb2bt2-mapping-2}
 \item HID usage IDをHID usage IDにマップする
  キーマッピングのテーブルを作成する \label{enum:usb2bt2-mapping-3}
\end{enumerate}

\mmmpart{手順\ref{enum:usb2bt2-mapping-1}}
loadkeysには読み込んだキーマッピングファイルを、キーコードをasciiコードに
マップするテーブルに変換して出力する機能がある。
本手順はこの機能を使うことで実現した\footnote{Linuxと同じく、
USB Host Shield 2.0とそのライブラリはデフォルトでは接続されたUSBキーボードの
キーレイアウトをusであると判断するので、loadkeysのシステムをそのまま使うことが
できた}。

しかし、loadkeysはデフォルトで、今回の用途では必要ない情報を出力したり、
逆に最低限のマッピングテーブルしか出力してくれなかったりした。
よって、その部分に関しては自分でloadkeysのプログラムを修正した。

本手順のために開発したプログラムを\cite{bib:keymap_table_gen}に示す。

\mmmpart{手順\ref{enum:usb2bt2-mapping-2}}
この手順を行うためには、キーコードをHID usage IDに変換するテーブルが必要だった。
このテーブルは自分で作成した。
キーコードもHID usage IDも、キーの印字とコードは一対一対応しているので、
このテーブルを作成するのは現実的であった。

このテーブルの作成は、以下の2つの方法で得たコードを照合させることで行った。
\begin{itemize}
 \item キーボードをコンピュータに接続し、コンピュータで
  showkey\footnote{コンソールで押下・解放されたキーボードのキーのキーコード
  を表示するプログラム。kbdに含まれている。}
  を起動させ、それぞれのキーのキーコードを読む。
 \item Arduinoに接続したUSB Host Shield 2.0にキーボードを接続し、
  ArduinoでそれぞれのキーのHID usage IDを読む。
\end{itemize}
今回作成したテーブルと、それをC言語から利用しやすい形式に変換する
プログラムを\cite{bib:keycode2hidusage}に示す。

\mmmpart{手順\ref{enum:usb2bt2-mapping-3}}
この手順を行うためには、asciiコードをHID usage IDに変換するテーブルが
必要である。
このテーブルとしては文献\cite{bib:uhl}のものを使用した。

また、今回は、kbdのキーマッピングの設定を上書きしたり、
コンシューマキーをマップするため、独自のフォーマットのマッピングファイルを
適用することができる仕組みも追加した。
これを使うことで、例えばAppleのiOSで検索のボックスを出すためのキーをF3キーに
割り当てたりすることができるようになる。

実際のマッピングのデータは数キロバイトに及んだ。
そのため、実行時にデータをメモリに展開せずフラッシュから値を直接読むPGMで
マッピングのデータを保存した。

\bigskip
完成した回路を図\ref{fig:usb2bt2-rn42-comp}に示す。

\begin{figure}[H]
\centering
\begin{minipage}{0.5\linewidth}
\frame{\includegraphics[height=\textheight,width=\textwidth,keepaspectratio]{\wrkdir img/usb2bt2-rn42-comp.png}}
\caption{完成した回路 (最上段:RN-42、最下段:Arduino)}
\label{fig:usb2bt2-rn42-comp}
\end{minipage}
\end{figure}
