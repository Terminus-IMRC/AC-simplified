\def \wrkdir {p.server/}

\mpart{Raspberry Piを用いたサーバの設計と構築と用途及び低レベルなライブラリの開発}
\label{pa:server}

私は自宅に Raspberry Pi によるサーバを構築した。
また、構築したサーバに新たな Raspberry Pi を接続し、
遠隔地に住む友人に計算資源として提供した。

\mmpart{ハードウェアにおける工夫}
Raspberry Pi は本来 SD カードから OS を起動させるが、
OS はログファイル等を SD カードに書き込むため、
SD カードが早く書き込み寿命に達してしまう。
よって、USB 経由でハードディスクを接続し、そこから OS を起動させるようにした。

また、Raspberry Pi の電源をデスクトップ用の電源から引くことにより、
サーバの動作の安定化を図った。

最終的なサーバの配線や回路の様子を図\ref{fig:srv-peri-circ1}と図\ref{fig:srv-peri-circ2}に示す。

\begin{figure}[H]
\centering
\begin{minipage}{0.45\linewidth}
\frame{\includegraphics[height=\textheight,width=\linewidth,keepaspectratio]{\wrkdir img/peri-circ1.png}}
\caption{サーバの配線と回路の様子 (ファン等の制御部)}
\label{fig:srv-peri-circ1}
\end{minipage}
\begin{minipage}{0.45\linewidth}
\frame{\includegraphics[height=\textheight,width=\linewidth,keepaspectratio]{\wrkdir img/peri-circ2.png}}
\caption{サーバの配線と回路の様子 (右:電源とその制御部、左上:Raspberry Pi 1、左下:Raspberry Pi 2)}
\label{fig:srv-peri-circ2}
\end{minipage}
\end{figure}
