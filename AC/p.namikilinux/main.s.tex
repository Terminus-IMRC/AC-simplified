\mpart{並列計算システムのための新たなLinuxディストリビューションの作成}
\label{pa:namikilinux}

第\ref{pa:ngk}\postmpartname{}で構築した並列計算システムは、
ノードやサービスの起動や終了等の操作が手動であった。
また、OSのサイズが大きく、ノードの起動が遅くなったうえ、
ノードはOS全体をメモリに保持するため、プログラムのメモリ不足が起こった。

その反省を踏まえ、操作を自動化し、かつサイズの小さい OS を作成した。
OS のベースとしては Tiny Core Linux \cite{bib:tc} を用い、
Linux カーネルやソフトウェアの再ビルドを行ったり、
操作を自動化するためのスクリプトを追加した。
